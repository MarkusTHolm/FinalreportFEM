\section*{Abstract}
\addcontentsline{toc}{section}{Abstract}

During the last 100 years, the development within the area of sandwich constructions has increased rapidly and new methods and knowledge within the field are repeatedly discovered. The constant investigations within the field allows for new optimized plate structures to be used in industries of high performance equipment, benefiting companies and global environment. Classical laminate and sandwich theory have played a major part in the development of this field and thus serves as the foundation in the results of this project.

This report visits the preliminary design of a helicopter sandwich floor, through the final report of the course \textit{41517 - Stiffened Plates and Sandwich Constructions}. With only few specifications regarding the geometrical constraints, a sandwich construction is analysed and developed, with the intention of being implemented in a NH90 Rescue/Transport helicopter. The development of the sandwich structure includes the material selection, design and manufacturing methods of a sandwich floor design with main criteria; low weight, long life, low cost in descending order of importance. At a maximum thickness of 30 mm, the sandwich construction must furthermore be able to withstand three specified load cases with a safety factor of maximum 1.8.

Through the design and analysis of this project a solution for the helicopter sandwich floor panels is presented. The final design comprises a 1.3mm thick Kevlar-Epoxy face sheet material with a 27.4mm thick aluminium expanded hexagonal honeycomb core with a density of \SI{50}{kg/m^3}. The Kevlar-Epoxy laminate face sheet consists of five plies following the lay-up $[90°,70°,-\overline{70°}]_s$. At a total weight, for all the floor panels, of 50.1 kg and a price total price of 4620\$, the proposed design fulfills the requirements and serves a prominent solution to the preliminary design of the helicopter sandwich floor. The report shows furthermore results, proving the efficiency and advantages of sandwich and laminate structures, presented through the analysis of different material selections and constructions. Using classical sandwich and laminate theory, results are attained similar to commercial finite element software, justifying the implementation of the classical theory. 



